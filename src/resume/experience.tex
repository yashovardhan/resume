\cvsection{Work Experience}
\begin{cventries}
	\cventry
	{Google Code-in Mentor 2017 \& 2018}
	{REDHAT JBoss Community}
	{Remote}
	{November 2017 – January 2018, \newline September 2018 - Present}
	{\begin{cvitems}
		\item {Mentor high school students trying to learn and contribute to open source projects with the JBoss Community}
		\item {Review their set tasks within a specified time limit.}
		\item {As a token of appreciation, I got a chance to attend the Google Code-in meet-up in the Google Asia-Pacific Headquarters, Singapore}
		\end{cvitems}}
		
	\cventry
	{Google Summer of Code Mentor \& Core Developer}
	{coala}
	{Remote}
	{February 2018 – September 2018}
	{\begin{cvitems}
		\item {One of the main developers of the open source organization.}
		\item {I had mentored the GSoC Project named 'Meta Review/ Git Task List' by Yana Agun Siswanto  under open source organization coala}
		\item {As a token for appreciation for my work, coala recommended my name for attending the GSoC Mentors' Summit in California}
		\item {Tech Stack: EmberJS, GitHub GraphQL API, GitLab API, Phabricator API, Wikidata Query API, Python, errbot, coala}
		\end{cvitems}}
		
	\cventry
	{React Native Developer}
	{Biller}
	{Jaipur, Rajasthan}
	{April 2018 – Present}
	{\begin{cvitems}
		\item {Main developer of the mobile User App and Waiter App for the Biller platform}
		\item {Managed the whole client side mobile application}
		\item {Made difference company specific components and sub modules incorporated into the applications}
		\item {The initial applications were built on Expo, which later on shifted to pure React Native.}
		\item {Tech Stack: React Native, Redux, Expo, Android \& iOS Development, NodeJS, Thunk, RestAPI, ESLint, TravisCI}
		\end{cvitems}}
		
    \cventry
	{Founder \& Maintainer}
	{Addy}
	{Jaipur, India}
	{September 2017 – Present}
	{\begin{cvitems}
		\item {Maintain the open source organization Addy ( \url{https://github.com/addy-org} )}
		\item {Maintain the wiki pages of Addy ( \url{http://addy.wiki} )}
		\item {This app became the inspiration for Google Plus Codes that works on a similar principle}
		\end{cvitems}}
		
\end{cventries}
